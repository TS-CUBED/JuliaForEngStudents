% Options for packages loaded elsewhere
\PassOptionsToPackage{unicode}{hyperref}
\PassOptionsToPackage{hyphens}{url}
\PassOptionsToPackage{dvipsnames,svgnames,x11names}{xcolor}
%
\documentclass[
  a4paper,
  DIV=11,
  numbers=noendperiod,
  oneside]{scrreprt}

\usepackage{amsmath,amssymb}
\usepackage{iftex}
\ifPDFTeX
  \usepackage[T1]{fontenc}
  \usepackage[utf8]{inputenc}
  \usepackage{textcomp} % provide euro and other symbols
\else % if luatex or xetex
  \usepackage{unicode-math}
  \defaultfontfeatures{Scale=MatchLowercase}
  \defaultfontfeatures[\rmfamily]{Ligatures=TeX,Scale=1}
\fi
\usepackage{lmodern}
\ifPDFTeX\else  
    % xetex/luatex font selection
\fi
% Use upquote if available, for straight quotes in verbatim environments
\IfFileExists{upquote.sty}{\usepackage{upquote}}{}
\IfFileExists{microtype.sty}{% use microtype if available
  \usepackage[]{microtype}
  \UseMicrotypeSet[protrusion]{basicmath} % disable protrusion for tt fonts
}{}
\makeatletter
\@ifundefined{KOMAClassName}{% if non-KOMA class
  \IfFileExists{parskip.sty}{%
    \usepackage{parskip}
  }{% else
    \setlength{\parindent}{0pt}
    \setlength{\parskip}{6pt plus 2pt minus 1pt}}
}{% if KOMA class
  \KOMAoptions{parskip=half}}
\makeatother
\usepackage{xcolor}
\usepackage[left=1in,marginparwidth=2.0in,textwidth=4.0in,marginparsep=0.3in]{geometry}
\setlength{\emergencystretch}{3em} % prevent overfull lines
\setcounter{secnumdepth}{-\maxdimen} % remove section numbering
% Make \paragraph and \subparagraph free-standing
\ifx\paragraph\undefined\else
  \let\oldparagraph\paragraph
  \renewcommand{\paragraph}[1]{\oldparagraph{#1}\mbox{}}
\fi
\ifx\subparagraph\undefined\else
  \let\oldsubparagraph\subparagraph
  \renewcommand{\subparagraph}[1]{\oldsubparagraph{#1}\mbox{}}
\fi


\providecommand{\tightlist}{%
  \setlength{\itemsep}{0pt}\setlength{\parskip}{0pt}}\usepackage{longtable,booktabs,array}
\usepackage{calc} % for calculating minipage widths
% Correct order of tables after \paragraph or \subparagraph
\usepackage{etoolbox}
\makeatletter
\patchcmd\longtable{\par}{\if@noskipsec\mbox{}\fi\par}{}{}
\makeatother
% Allow footnotes in longtable head/foot
\IfFileExists{footnotehyper.sty}{\usepackage{footnotehyper}}{\usepackage{footnote}}
\makesavenoteenv{longtable}
\usepackage{graphicx}
\makeatletter
\def\maxwidth{\ifdim\Gin@nat@width>\linewidth\linewidth\else\Gin@nat@width\fi}
\def\maxheight{\ifdim\Gin@nat@height>\textheight\textheight\else\Gin@nat@height\fi}
\makeatother
% Scale images if necessary, so that they will not overflow the page
% margins by default, and it is still possible to overwrite the defaults
% using explicit options in \includegraphics[width, height, ...]{}
\setkeys{Gin}{width=\maxwidth,height=\maxheight,keepaspectratio}
% Set default figure placement to htbp
\makeatletter
\def\fps@figure{htbp}
\makeatother
\newlength{\cslhangindent}
\setlength{\cslhangindent}{1.5em}
\newlength{\csllabelwidth}
\setlength{\csllabelwidth}{3em}
\newlength{\cslentryspacingunit} % times entry-spacing
\setlength{\cslentryspacingunit}{\parskip}
\newenvironment{CSLReferences}[2] % #1 hanging-ident, #2 entry spacing
 {% don't indent paragraphs
  \setlength{\parindent}{0pt}
  % turn on hanging indent if param 1 is 1
  \ifodd #1
  \let\oldpar\par
  \def\par{\hangindent=\cslhangindent\oldpar}
  \fi
  % set entry spacing
  \setlength{\parskip}{#2\cslentryspacingunit}
 }%
 {}
\usepackage{calc}
\newcommand{\CSLBlock}[1]{#1\hfill\break}
\newcommand{\CSLLeftMargin}[1]{\parbox[t]{\csllabelwidth}{#1}}
\newcommand{\CSLRightInline}[1]{\parbox[t]{\linewidth - \csllabelwidth}{#1}\break}
\newcommand{\CSLIndent}[1]{\hspace{\cslhangindent}#1}

\usepackage{amsmath, amssymb}
\KOMAoption{captions}{tablesignature}
\makeatletter
\makeatother
\makeatletter
\makeatother
\makeatletter
\@ifpackageloaded{caption}{}{\usepackage{caption}}
\AtBeginDocument{%
\ifdefined\contentsname
  \renewcommand*\contentsname{Table of contents}
\else
  \newcommand\contentsname{Table of contents}
\fi
\ifdefined\listfigurename
  \renewcommand*\listfigurename{List of Figures}
\else
  \newcommand\listfigurename{List of Figures}
\fi
\ifdefined\listtablename
  \renewcommand*\listtablename{List of Tables}
\else
  \newcommand\listtablename{List of Tables}
\fi
\ifdefined\figurename
  \renewcommand*\figurename{Figure}
\else
  \newcommand\figurename{Figure}
\fi
\ifdefined\tablename
  \renewcommand*\tablename{Table}
\else
  \newcommand\tablename{Table}
\fi
}
\@ifpackageloaded{float}{}{\usepackage{float}}
\floatstyle{ruled}
\@ifundefined{c@chapter}{\newfloat{codelisting}{h}{lop}}{\newfloat{codelisting}{h}{lop}[chapter]}
\floatname{codelisting}{Listing}
\newcommand*\listoflistings{\listof{codelisting}{List of Listings}}
\makeatother
\makeatletter
\@ifpackageloaded{caption}{}{\usepackage{caption}}
\@ifpackageloaded{subcaption}{}{\usepackage{subcaption}}
\makeatother
\makeatletter
\@ifpackageloaded{tcolorbox}{}{\usepackage[skins,breakable]{tcolorbox}}
\makeatother
\makeatletter
\@ifundefined{shadecolor}{\definecolor{shadecolor}{rgb}{.97, .97, .97}}
\makeatother
\makeatletter
\makeatother
\makeatletter
\@ifpackageloaded{sidenotes}{}{\usepackage{sidenotes}}
\@ifpackageloaded{marginnote}{}{\usepackage{marginnote}}
\makeatother
\makeatletter
\makeatother
\ifLuaTeX
  \usepackage{selnolig}  % disable illegal ligatures
\fi
\IfFileExists{bookmark.sty}{\usepackage{bookmark}}{\usepackage{hyperref}}
\IfFileExists{xurl.sty}{\usepackage{xurl}}{} % add URL line breaks if available
\urlstyle{same} % disable monospaced font for URLs
\hypersetup{
  colorlinks=true,
  linkcolor={blue},
  filecolor={Maroon},
  citecolor={Blue},
  urlcolor={Blue},
  pdfcreator={LaTeX via pandoc}}

\author{}
\date{}

\begin{document}
\ifdefined\Shaded\renewenvironment{Shaded}{\begin{tcolorbox}[breakable, interior hidden, borderline west={3pt}{0pt}{shadecolor}, frame hidden, boxrule=0pt, enhanced, sharp corners]}{\end{tcolorbox}}\fi

\hypertarget{julia-features-for-scientic-computing}{%
\chapter{Julia Features for Scientic
Computing}\label{julia-features-for-scientic-computing}}

While the question ``What's the best programming language?'' is
impossible to answer\sidenote{\footnotesize and the answer will be different for
  different applications.}, Julia has some specific features that make
it particularly well suited for scientific computing:

\begin{itemize}
\item
  \textbf{just-in-time (JIT) compilation}, which allows performance that
  is on the same level as Fortran or C code.
\item
  \textbf{multiple dispatch}, a programming paradigm, which, while a
  functional paradigm, brings some of the possibilities usually
  associated with object-oriented (OO) programming\sidenote{\footnotesize Julia is not
    an OO language, which some may see as a disadvantage. Personally I
    never found OO to be useful for the kind of scientific programming I
    do.}.
\item
  \textbf{1-based indexing}\sidenote{\footnotesize Julia supports arbitrary indexing,
    but 1-based is the standard.}, which makes the code more closely
  resemble much of the mathematical concept we want to
  express\sidenote{\footnotesize Some, coming from C or Python may see 1-based
    indexing as a negative.}.
\item
  \textbf{Matlab similar syntax}: coming from Matlab, Julia's syntax
  will be familar enough to allow a quick transition\sidenote{\footnotesize My first
    experience with Julia was, indeed, porting an ODE problem from
    Matlab to Julia. And most of the work was replacing two function
    calls and switching square for round brackets.}.
\end{itemize}

\hypertarget{just-in-time-compilation-and-the-two-language-problem}{%
\section{\texorpdfstring{Just-In-Time Compilation and the \emph{Two
Language
Problem}}{Just-In-Time Compilation and the Two Language Problem}}\label{just-in-time-compilation-and-the-two-language-problem}}

Many scientific models are developed in \emph{prototyping} languages
like Python (or Matlab), since they are interactive, i.e., they run in
an ``read-eval-print-loop'' (REPL), which offers immediate feedback and
results, encouraging experimentation and ``playing'' with the code. This
is a key feature for many scientists, as opposed to software developers,
since it allows for quick development of a working model.

The problem with these interactive languages is, that they tend to be
\emph{interpreted}, and this much slower than \emph{compiled} languages.
We will see later that, e.g., Python can be up to 3 orders of magnitude
slower than a compiled language, which also has huge implications on
energy efficiency of scientific computing, as a contributor to climate
change (Pereira et al. 2017,
2021)\marginpar{\begin{footnotesize}\leavevmode\vadjust pre{\protect\hypertarget{ref-Pereira2017}{}}%
Pereira, Rui, Marco Couto, Francisco Ribeiro, Rui Rua, Jácome Cunha,
João Paulo Fernandes, and João Saraiva. 2017. {``Energy Efficiency
Across Programming Languages: How Do Energy, Time, and Memory Relate?''}
In \emph{Proceedings of the 10th {ACM SIGPLAN International Conference}
on {Software Language Engineering}}, 256--67. {SLE} 2017. {New York, NY,
USA}: {Association for Computing Machinery}.
\url{https://doi.org/10.1145/3136014.3136031}.\vspace{2mm}\par\end{footnotesize}}\marginpar{\begin{footnotesize}\leavevmode\vadjust pre{\protect\hypertarget{ref-Pereira2021}{}}%
---------. 2021. {``Ranking Programming Languages by Energy
Efficiency.''} \emph{Science of Computer Programming} 205 (May): 102609.
\url{https://doi.org/10.1016/j.scico.2021.102609}.\vspace{2mm}\par\end{footnotesize}}.

So with any interpreted language scientific computing will arrive at the
point where the performance of the developed code is no longer
sufficient. Usually the next step in these cases is either

\begin{enumerate}
\def\labelenumi{\alph{enumi}.}
\item
  abandonment of the model.
\item
  re-implementation of the model in a compiled language like Fortran or
  C.
\end{enumerate}

This is what is called the \textbf{``Two Language Problem''}.

The two-language-problem is particularly concerning for scientific
computing, since most engineers and natural scientists are not
computer-scientists or software developers, or coders. Many of us do not
readily speak a compiled language which would require writing code on a
much lower abstraction level than most of us are comfortable
with\sidenote{\footnotesize I do know how to write Fortran, but it is a much more
  daunting prospect than doing the same thing in Julia (or Matlab or
  Python).}.

Julia can overcome this problem by virtue of being ``just-in-time''
compiled. This means that on the surface it looks like an interpreted
language, while, behind the scenes and completely transparent to the
user, the code is compiled to machine code and then executed. The Julia
developers coined the phrase:

\begin{quote}
``Feels like Python, runs like C.''
\end{quote}

With only little effort, we can write Julia code that runs at the same
speed as optimised Fortran or C code\sidenote{\footnotesize In fact, Julia is the only
  dynamically typed language that has managed to
  \href{https://www.hpcwire.com/off-the-wire/julia-joins-petaflop-club/}{join
  the exclusive Petaflop Club} of peak performance of greater than one
  petaflops (1015 floating point operations per second), scaling to over
  1 million threads.}.




\end{document}
